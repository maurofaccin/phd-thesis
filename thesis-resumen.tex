%%%%%%%%%%%%%%%%%%%%%%%%%%%%%%%%%%%%%%%%%%%%%%%%%%%%%%%%%%%%
\chapternonum{Resumen}
\nopagebreak
%%%%%%%%%%%%%%%%%%%%%%%%%%%%%%%%%%%%%%%%%%%%%%%%%%%%%%%%%%%%




%\paragraph{} %senza questo non funziona il correttore ortografico nel resto
\lettrine{E}{l} estudio de esta Tesis se centra en de dos problemas
biológicos distintos aplicando un enfoque estadístico-mecánico común:
en el primer caso el trabajo se centra en el problema de la secuenciación de los
péptidos a través de la interpretación \emph{de novo} de espectros de masa
tándem, mientras que en el segundo se estudia el plegamiento de la Miotrofina,
una pequeña proteína de tipo modular que ha sido recientemente caracterizada por
diferentes técnicas experimentales.
 
Ambos problemas, como la mayoría de los problemas de tipo biológico, son
demasiado complejos como para ser tratados de una manera detallada, \emph{ab
initio}, basándose en una descripción microscópica de las interacciones y de
los procesos fundamentales involucrados; generalmente para poderlos tratar, se tiene que introducir
varias simplificaciones en el modelo del sistema original, a costa de
incluir, inevitablemente, ciertas aproximaciones.
El típico enfoque empleado en la descripción de sistemas biológicos desde un
punto de vista mecánico-estadístico se basa en la traducción efectiva del
sistema complejo original a un modelo físico tratable, posiblemente involucrando
el uso de coarse-graining u otros tipos de simplificaciones.

En el caso tratado en esta Tesis hemos reescrito el problema original mencionado antes, en
términos de un oportuno sistema mecánico estadístico a través de la definición
de las variables dinámicas interesantes y de una función energía que determina
su comportamiento.
Además en ambos problemas hemos forzado una modelización de las interacciones que
nos permita calcular una solución exacta de la distribución de probabilidad en el
equilibrio a través del método de la matriz de transferencia.

Ambos problemas están relacionados con proteínas: su identificación con MS/MS y su
plegamiento. 


%esta Tesis tratamos de caracterizar dos aparentemente distintos problemas
%biol\'ogicos: en el primer caso nos acercamos al problema del secuenciación de
%p\'eptidos en el campo de la Espectrometr\'ia de Masa, mientras en el segundo hemos
%analizado el plegamiento del la prote\'ina modular Miotrofina.
%La transformaci\'on de los problemas biol\'ogicos a un sistema mec\'anico estad\'istico
%se ha tratado con la definici\'on de las variables din\'amicas interesantes y con la
%escritura de una funci\'on conste que gobierna sus comportamientos.
%El sistema, si separable en subsistemas con interacciones solamente a primeros
%vecinos, ofrece una soluci\'on exacta por la distribuci\'on de probabilidad al
%equilibrio a trav\'es del mecanismo de la matriz de transferencia.
%En este estudio nos basamos en esta caracter\'istica para profundizar en la
%descripci\'on de estos sistemas y su diferentes conformaciones.

%\paragraph{¿Que son las Proteínas?}
%Las Proteínas son moléculas omnipresentes en las células, ellas son involucradas
%en la mayoría de los procesos bioquímicos que gobiernan su evolución.
%Las proteínas son polímeros compuestos por secuencias de aminoácidos, y cuya
%estructura puede ser descrita por medio de una sucesión de símbolos que componen
%un alfabeto de aminoácidos, esta sucesión se llama estructura primaria.
%La cadena de aminoácidos que compone la proteína se pliega generalmente en una
%dada, usualmente globular, estructura tridimensional, adquiriendo de esta forma
%la función bioquímica que tendrá que desempeñar.
%
%La síntesis de proteínas se lleva a cabo dentro de la célula a través de un
%mecanismo de traducción del ADN.
%La información de la composición en aminoácidos es, en efecto, codificada en la
%doble hélice del ADN. Éste se encuentra empaquetado en el nucleo de la célula
%donde, si hay demanda, se desempaqueta localmente para transcribir su contenido
%en un suporte temporal llamado ARN.  
%El ARN sale del núcleo de la célula y entra en el citoplasma donde se encuentran
%los Ribosomas, aquí el ARN puede someterse a algunas modificaciones antes de
%empezar la traducción a una cadena polipeptídica.
%Los Ribozomas descodifican el ARN mensajero usando unas reglas que asignan a
%cada tripleta de nucleótides del ARN un residuo especifico, este, entonces,
%es agregado a la cadena del péptido final.
%
%Una vez concluida la construcción de la cadena de aminoácidos, esta entra en el
%proceso de plegamiento para alcanzar la estructura tridimensional que le
%confiere su función bioquímica.
%A menudo a este estadio de la síntesis, la proteína puede subir unas
%modificaciones, caracterizadas sobre todo por agregación o perdida de grupos
%funcionales.
%
%Las proteínas se pliegan en una estructura estable, donde se encuentran a menudo
%unos patrones dominantes, sobre todo hélices $\alpha$ y hojas $\beta$, llamadas
%estructuras secundarias.
%Estos módulos se ordenan en la estructura terciaria global de la molécula
%plegada.
%Una estructura cuaternaria se puede encontrar en los complejos de proteínas
%ligadas entre si a formare una superestructura funcional.
%
%Las proteínas se pliegan de manera natural en las condiciones biológicas a las
%que son sometidas, experimentando un proceso de plegamiento fuera del
%equilibrio, desde una configuración priva de estructura, la proteína alcanza el
%estado nativo.
%El plegamiento es caracterizado por el camino seguido por el polímero en este
%proceso que describe la formación, dependiente del tiempo, de los contactos a
%largo alcance entre residuos no vecinos.
%El principal mecanismo de este proceso puede ser resumido en cuatro
%comportamientos principales\cite{Nickson2010}: 
%los módulos de estructura secundaria se forman primero, luego estos difunden y
%chocan hasta formar la estructura correcta; los residuos hidrófobos colapsa en
%centro como consecuencia del ambiente acuoso, provocando el plegamiento; un
%núcleo central empieza el proceso, propagándose luego a resto de la cadena; el
%núcleo de agregación no basta a provocar el plegamiento, este adviene cuando un
%numero critico de enlaces a largo alcance se hayan formado.
%
%La caracterización del camino de plegamiento se basa en la definición de los rasgos de
%cada conformación molecular antes de alcanzar el estado nativo\cite{Daggett2002}
%que refleja las características del paisaje energético.
%Esto ha sido alcanzado de manera satisfactoria solamente en proteínas pequeñas,
%combinando resultados experimentales con simulaciones de dinámica molecular.
%De otra parte, estudios experimentales de plegamiento de proteínas proporciona
%una instantánea de la estructura a lo largo del camino que, aunque no
%proporciona una información temporal de la formación de contactos, puede sugerir
%los atributos de los estados intermedios y del mecanismo general del proceso.

%%%%%%%%%%%%%%%%%%%%%%%%%%%%%%%%%%%%%%%%%%%%%%%%%%%%%%%%%%%%%%%%%%%%%%%%%%
\paragraph{Secuenciamiento de Prote\'inas}

En la primera parte de la Tesis afrontamos el problema de la secuenciación de
p\'eptidos en el marco de la Espectrometr\'ia de Masa, que consiste en la
interpretación de un espectro de masa para encontrar la secuencia de amino
ácidos del péptido estudiado.
Gracias a su simplicidad y bajo coste, el uso de esta herramienta se
utiliza ampliamente en el campo del análisis bioquímica de muestras
desconocidas de proteínas, y, generalmente, está incrustada en una
instrumentación de tipo high-throughput automatizada, que produce una gran
cantidad de datos, necesitando entonces una herramienta automatizada para
interpretarlos. 

En principio, un espectro de masa tándem contiene toda la información necesaria
para encontrar la secuencia de amino ácidos del péptido que generó el espectro mismo.
En la práctica, encontrar dicha secuencia desde un espectro es una tarea muy
difícil por la presencia de ruido, picos debidos a iones
contaminantes, o falta de fragmentaciones, entre otras cosas.
De echo, cada espectro es el resultado de las reglas microscópicas que gobiernan
la transferencia de energía y la fragmentación estocástica del péptido precursor
en las colisiones, en presencia de ruido y contaminantes de diferentes tipos.
Desafortunadamente, la predicción \emph{ab initio} del espectro resultante a
partir de la secuencia del péptido precursor es impracticable, si no imposible,
y la interpretación de la secuencia del péptido involucra el uso de funciones
coste \emph{ad hoc} para medir el acuerdo entre el espectro teórico de un
péptido precursor y el espectro experimental.
Además, el espacio de búsqueda de las secuencias es generalmente limitada a las
secuencias de proteínas ya conocidas (``búsqueda sobre bases de datos''). 
Gracias a esta restricción la búsqueda de la secuencia resulta
más practica y eficiente de los métodos \emph{de novo}
que infieren la secuencia del péptido a partir solo de la información contenida
en el espectro, pero está afectada por sus limitaciones.
Finalmente en ambos métodos es importante subrayar que un problema central es la
asignación de un ``grado de confianza'' a las predicciones, entre las cuales
se pueden encontrar falsos positivos, o sea secuencias equivocadas con alta
puntuación.


A continuación describimos un nuevo algoritmo basado en la traducción del problema de la
interpretación, al análisis de la distribución al equilibrio de un sistema
físico discreto adecuadamente definido, cuyas variables dinámicas describen la
presencia de un enlace peptídico (un sitio de fragmentación) en los nodos de un
retículo unidimensional, etiquetado por un índice de masa.
La función energía \emph{ad hoc} que gobierna el modelo, se deduce de la
distribución fenomenológica de los iones y de los picos de ruido en un conjunto de espectros
experimentales.
Las interacciones que caracterizan esta Hamiltoniana son interacciones locales o
a primeros vecinos de manera que la función de partición del modelo puede ser
calculada exactamente con el método de la matriz de transferencia.
Mientras que la identificación de la secuencia del péptido está asociada a la
caracterización del estado fundamental a temperatura cero, la introducción de
una temperatura paramétrica y el estudio de las variables termodinámicas como
función de la temperatura pueden dar una idea de la calidad de la
interpretación, sin apoyarse a bases de datos decoy (bases de datos compuestas por
secuencias equivocadas pero con alta puntuación) o en la distribución
fenomenológicas de las puntuaciones de los falsos positivos.

Aumentando la temperatura, el equilibrio del sistema se aleja
de un régimen dominado por la energía y polarizado hacía la secuencia mejor
puntuada, para alcanzar un régimen dominado por la entropía debida a otras
secuencias con una menor puntuación, lo cual puede ser útil para evaluar la bondad de la predicción.

El algoritmo ha sido testado sobre un conjunto de espectros experimentales 
acoplados con la secuencia teórica del péptido precursor y sobre el mismo
conjunto han sido testados algunos de los algoritmos \emph{de novo} disponibles
(NovoHMM, PepNovo, Lutefisk).
Nuestro algoritmo produce resultados comparables con los programas existentes
y además exhibe algunas características útiles relacionadas con el sistema 
termodinámico asociado que no se encuentran en los demás.
Sobresale como característica del algoritmo
la posibilidad de controlar la temperatura de trabajo que,
junto con la posibilidad de calcular exactamente la distribución de
probabilidad al equilibrio, proporciona la oportunidad de considerar al
mismo tiempo el entero espacio de las secuencias cada una con su ``peso''
termodinámico.

%%%%%%%%%%%%%%%%%%%%%%%%%%%%%%%%%%%%%%%%%%%%%%%%%%%%%%%%%%%%%%%%%%%%%%%%%%%%%%
\paragraph{Plegamiento de la Prote\'ina Miotrofina}
En la segunda parte de la Tesis nos hemos centrado en el problema del
plegamiento del las proteínas, tratando de caracterizar en particular el
equilibrio y la cinética de la Miotrofina, una proteína de tipo modular.
El plegamiento de proteínas es un problema que ha sido enormemente estudiado y
analizado por parte de los teóricos a partir de muchos niveles de
coarse-graining, empezando por
las simulaciones a todos los átomos con interacciones realistas, a una variedad
de modelos de coarse-graining.

La proteína estudiada en este trabajo, la Miotrofina, es una proteína modular
compuesta por cuatro módulos con la misma estructura secundaria aunque
diferentes secuencias, dispuestos en una conformación lineal.
Esta conformación, común a las proteínas modulares, las diferencia de las ampliamente estudiadas
proteínas globulares. 
En las proteínas globulares las
regiones alejadas en la secuencia se acercan y entran en ``contacto'' cuando la
proteína se encuentra en el estado nativo, y el alcance de los contactos puede
abarcar la entera molécula.
En las proteínas modulares solo hay contactos dentro de cada
módulo y entre amino ácidos pertenecientes a módulos contiguos.
La Miotrofina muestra un comportamiento interesante que ha atraído la atención
de los investigadores: la molécula parece plegarse de manera cooperativa,
característica típica de las proteínas globulares, mientras su estructura
modular sugiere una independencia intrínseca de cada módulo, naturalmente
asociado a un plegamiento con varios estados
intermedios que corresponden al plegamiento independiente de los módulos.


En este trabajo nos basamos en el modelo propuesto por Wako- Saito-Mu\~noz-Eaton
(WSME) 
para caracterizar el plegamiento de la Miotrofina.
Este simple modelo se basa en el uso de una variable binaria para describir el estado
(``plegado'' o ``desnaturalizado'') de cada residuo usando la información
contenida en la estructura nativa de la molécula, la cual está almacenada en un mapa de
contactos que describe la proximidad euclídea de dos residuos y de esta forma predecir el
comportamiento en el proceso de plegamiento.
A pesar de la presencia de interacciones a largo alcance, la forma especial de la función energía
del modelo, le otorga la posibilidad de evaluar exactamente la
función de partición, así como las cantidades termodinámicas en el equilibrio, como
la energía libre y la capacidad calorífica, mientras que para describir la dinámica
del sistema hay que usar unas simulaciones de tipo Monte Carlo.

Usamos el modelo WSME para estudiar la molécula wild-type como algunas
mutaciones y por esto tratamos las mutaciones puntuales como variaciones del potencial
de interacción del residuo interesado.

En ambos casos, los resultados muestran un buen acuerdo con los datos
experimentales, haciendo particular hincapié en la heterogeneidad de los caminos de
plegamiento.
El minucioso control sobre las simulaciones, que sobrepasa las posibilidades
experimentales, nos otorga la posibilidad de sugerir las caracter\'isticas de los
procesos subyacentes, entre ellos sobresale el acuerdo entre la estructura modular, 
que produce un
perfil de la energía libre con múltiple mínimos, y la cooperaci\'on en el proceso
de plegamiento, y además sobresale que la simetr\'ia entre los caminos de
plegamiento y de desnaturalizaci\'on de la mol\'ecula no es una condición necesaria.



