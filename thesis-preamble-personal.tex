
% Aqu� puedes a�adir otros paquetes que quieras incluir... Ojo deben ser
% cosas que NO modifiquen el estilo b�sico del libro.

% [Nota: en este fichero todas las l�neas est�n en blanco o comentadas
% (comienzan por %). Si quieres descomentar algo no tienes m�s que
% eliminar el signo % al comienzo de la l�nea.


% Algunos ejmeplos de paquetes m�s o menos habituales son...
% \usepackage{bm}
% \usepackage{ae, aecompl, aeguill}
% \usepackage{wrapfig}
% \usepackage[dvips]{color}
% \usepackage{rotating}
% \usepackage{subfigure}
% Existen muchos m�s. Elige los que necesitas y �salos. En la red
% encontraras informaci�n abundante al respecto.

%%%%%%%%%%%%%%%%%%%%%%%%%%%%%%%%%%%%
%%%%%%%%% MIS PAQUETES  %%%%%%%%%%%%
%%%%%%%%%%%%%%%%%%%%%%%%%%%%%%%%%%%%


\usepackage{wrapfig}
% wrapfig modifiers
\newlength{\wrapsep}
\setlength{\wrapsep}{10mm}
\newlength{\saveintextsep}
\setlength{\saveintextsep}{\intextsep}
\newlength{\wrapsepcol}
\setlength{\wrapsepcol}{10mm}
\newlength{\savecolumnsep}
\setlength{\savecolumnsep}{\columnsep}


% Sideways of figures & tables
\usepackage{rotating}
% Labels subfigures within a figure environment.
\usepackage{subfigure}

\usepackage{tikz}

\usepackage[square,comma,numbers,sort&compress]{natbib}  
\usepackage[breaklinks]{hyperref}

%%%%%%%%%%%%%%%%%%%%%%%%%%%%%%%%%%%%
%%%%%%%%% FIN PAQUETES  %%%%%%%%%%%%
%%%%%%%%%%%%%%%%%%%%%%%%%%%%%%%%%%%%

\newcommand{\enne}{\mathcal{N}}

\newcommand{\ournovo}{\mbox{\emph{T-novoMS}}}
\DeclareMathOperator*{\argmax}{arg\,max}

\newcommand{\sectionnonum}[1]{
\phantomsection
\addcontentsline{toc}{section}{#1}  % to add the section to the Contents
\section*{#1}                       % to add a un-numbered section
}
\newcommand{\chapternonum}[1]{
\phantomsection
\addcontentsline{toc}{chapter}{#1}  % to add the section to the Contents
\chapter*{#1}                       % to add a un-numbered section
}
\newcommand{\partnonum}[1]{
\phantomsection
\addcontentsline{toc}{part}{#1}  % to add the section to the Contents
\part*{#1}                       % to add a un-numbered section
}


% Hypothesis and theorems
\newtheorem{hypothesis}{Hypothesis}[chapter]


% Numeration for introduction figures:
\newcounter{partintro}[part]

\hyphenation{Schr\"o-din-ger}
\hyphenation{me-ta-bo-lic}
\hyphenation{be-hav-iours}
