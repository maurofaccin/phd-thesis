\chapter{Constraints in the state variables}
\label{chap:app_constraints}

In this appendix we describe the constraints applied to the state-variables
$q_\nu^X,\bm l_\nu^X$ and $\pi_\nu$ presented in Section~\ref{sec:variables}.

A fragmentation site in $\nu$ is characterized by $r_\nu=0$; the  values of the dynamical variables
$\sigma_{\nu-1}$ and $\sigma_\nu$ at the sites $\nu-1$ and $\nu$ specify the nature of the residue N-terminal to the fragmentation sites (the ``last completed residue'').
% If the residue $a$ ends in $\nu$, then the fragmentation of the peptide 
The  fragmentation at $\nu$ will produce a number of ions that depend on $\sigma_{\nu-1}$ and $\sigma_\nu$. 
% residues accumulated on the N-term and on the C-term.

We define the ``maximal effective charge'', N- or C- terminal to the fragmentation point, 
$q^X_\nu$ ($X=N,C$), as:
\begin{equation}
q^X_\nu = \min(Q-1,n_X)
\end{equation}
with $Q$ the total charge of the parent peptide as detected by the first mass spectrometer and $n_N$, $n_C$ the number of basic residues K, R, H, contained in the N- or C- fragment, respectively.
In practice, $q^X_\nu \in [0,Q-1]$ counts the number of basic residues $X$-terminal to $\nu$, until this number reaches the maximal possible value, which is $Q-1$ (since a unit charge is always present from the extra proton attached to the parent peptide) 
In the fragmentation process, $q^X_\nu$ imposes an upper limit on the number of charges species that can be generated:
for instance, $q^N_\nu =0$ and $q^C_\nu =2$ at some $\nu$  for a triply charged spectrum will imply that the N-fragments  generated at $\nu$ can just have charge $+1$, while the C-fragments can have charge in the range $[1,3]$. 

The constraints for the values at neighboring sites are as follows:
\begin{itemize}
 \item If  $r_\nu \ne 0$, that is, if $\nu$ does not correspond to a peptide bond (a fragmentation sites), or if $r_\nu = 0$ but the residue $a$ ending in $\nu$ is not H,K or R, then
 $q^X_\nu=q^X_{\nu-1}$  (that is $\delta_q^X(a)=0$) it the only admitted possibility, for both $X=N,C$.

\item if $r_\nu = 0$ and the residue $a$ N-terminal to $\nu$ is  H,K or R, then
$q^N_\nu=q^N_{\nu-1}+1$ (and $\delta_q^N(a)=1$), provided that $q^N_{\nu-1}<Q-1$, otherwise, 
if $q^N_{\nu-1}=Q-1$, $q^N_\nu=q^N_{\nu-1}$.
For the C-term fragments, if $r_\nu = 0$ and the residue $a$ N-terminal to $\nu$ is  H,K or R, then
$q^C_\nu=q^C_{\nu-1}-1$ ($\delta_q^C(a)=-1$), unless $q^C_{\nu-1}=Q-1$, in which case also  $q^C_\nu=q^C_{\nu-1}$ is allowed, in addition to the already mentioned rule.
\end{itemize}
The rule on $q^C_\nu$ allows to deal with the fact that the number of basic residues C terminal to a point is not known  a priori, and not necessarily all of them will be charged.


% When $q^N_{\nu-1}$ has reached the maximum expected charge value, represented by
% the precursor charge state $Q$, then the number of observable charges 
% in the experimental peaks cannot
% increase furthermore and $q^N_\nu=q^N_{\nu-1}=Q$ for every residue $a$.
% The C-term charge content depends on the residue content after 
% the fragmentation site $\nu$. This is unknown and unpredictable so the 
% interaction involving $q^C_\nu=q^C_{\nu-1}=Q$ is allowed.

The constraints on the neutral losses, 
relating  the values of $\bm l^X_{\nu-1}$ and $\bm l^X_\nu$, 
follow
analogous rules of those applied to the charge.
If the appended residue $a$ belongs to the set of residues that can 
loose the neutral group $i$, then $l^N_{\nu,i}=l^N_{\nu-1,i}+1$
and $l^C_{\nu,i}=l^C_{\nu-1,i}-1$, otherwise $l^X_{\nu,i}=l^X_{\nu-1,i}$.
Moreover, as in the charge case, a further rule is introduced to
respect the maximum expected neutral loss values $\bm l_\text{max}$. 
In this case the latter are  parameters of the algorithm, and are not provided together with the parent spectrum.  
%and depend on which and how many ion types are matched to the experimental spectrum.
The additional rules affects the case in which %imply that the presence of interaction if 
$
%l^X_{\nu,i}=
l^X_{\nu-1,i}=l_{\text{max},i}$. 
and are completely analogous to the  rules for the charge,  above.


A further interaction constraint is introduced to reproduce
the behaviour of the Trypsin digestion. The outcoming 
precursor peptides, product of the digestion of the target protein
by the Trypsin enzyme, follow a common pattern.
Trypsin cleaves the protein at the carboxyl side of the residues
Lysine (K) and Arginine (R),
while the cleavage is inhibited if the following residue is either a 
Proline (P) or another K or R.
We distinguish, then, between three types of residue: the cleaving residues (K and R),
the residues that prevent a previous cleavage (P,K and R), and the rest
of residues. 
The value of $\pi_\nu$ define the nature of the previous residue and force
the following residue to respect the tryptic rules: a value $\pi_\nu=1$  means that
the previously added residue was a cleaving one, so that  a P,K or R
are expected at the next $\nu$ where $r_\nu=0$, unless it is the end of the chain..

The following rules implement the correct constraints:
\begin{itemize}
 \item if $r_\nu \neq 0$, $\pi_\nu=\pi_{\nu-1}$;
\item if $r_\nu = 0$ and $a$ is the species of the residue ending at $\nu$, then 
\begin{itemize}
\item if $\pi_{\nu-1}=0$ and $a$ is not K or R, then $\pi_\nu=0$,
\item if $\pi_{\nu-1}=0$ and $a$ is  K or R, then $\pi_\nu=1$,
\item if $\pi_{\nu-1}=1$ and $a$ is  K or R, then $\pi_\nu=1$,
\item if $\pi_{\nu-1}=1$ and $a$ is P, then $\pi_\nu=0$.
\end{itemize}
\end{itemize}
Every other combination is forbidden.
% In the rest of the cases $\pi_{\nu-1}=\pi_\nu=0$ allow the interaction ($\delta_\pi(a)=0$).
% This rule is not observed when a cleavage residue (either K or R) ends
% in $\nu$, forcing $p_\nu=1$ ($\delta_\pi(a)=1$).
% When $\pi_{\nu-1}=1$ then only the residues that prevent the cleavage are
% allowed (P, K or R), but while Proline allows to continue the normal
% behaviour with $\pi_\nu=0$ ($\delta_\pi(a)=-1$), if Lysine or Arginine are appended
% they still need to be followed by a residue that prevent the cleavage
% with $\pi_\nu=1$ ($\delta_\pi(a)=0$).

The ions generation variables $\xi^{a}_\nu$ only depend on the local state and
are non zero if and only if $r_\nu=0$ and the ion species considered are
compatible with the local charge and neutral losses states (i.e. $\xi^{y^{++}}_\nu$
can be 1 only if $q^C_\nu\ge 1$, %$q^C_\nu\ge2$
 analogously $\xi^{b-NH_3}_\nu$ can take value 1
if $l^N_{\nu,NH_3}\ge1$).

In some cases may be useful to keep track of the the number of residues accumulated from the N-term, $n_\nu$. In that case, the constraint is obvious: $n_\nu=n_{\nu-1}+1$ if $r_nu=0$, otherwise $n_\nu=n_{\nu-1}$. 


\paragraph{Boundaries Conditions.}
At the N-terminal, at $\nu=0$, and at the C-terminal, $\nu=M$, the boundaries
conditions of the system are characterized by the values of
the variables $\sigma_0$ and $\sigma_M$ that correspond to an extremity for the
residue sequence. 
At N-term the
first residue starts at $\nu=0$ with $r_0=0$, $n_0=1$,  and all residue types are admitted
and then 
$\pi_0=1,0$; $q^N_0=0, \bm l^N_0=\bm 0$, while
C-terminal ions can express with the maximum of the charge, $q^C_0=Q-1$, and, as we
do not know the maximum number of neutral losses because we ignore the peptide sequence,
then $l^C_{0,i}=1$ for every $i$ in $[0,l_{\text{max},i}]$.
On the contrary at C-terminal we have $r_M=0$ but $n_M$.
Analogously to the N-terminal boundaries, here we have $q^C_M=0$ and $\bm
l^C_M=\bm 0$; and for the N-terminal ions we have $q^N_M=Q-1$ and
$l^N_{M,i}=1$ for $i\in[0,l_{\text{max},i}]$.

