\chapternonum{Conclusiones}

Concluyendo, se resumen los resultados más importantes descritos en esta
Tesis y se presentan algunas propuestas para estudios futuros.
En esta Tesis hemos tratado la aplicación de métodos analíticos y computacionales a
problemas de tipo biológico, y hemos mostrado que, si este último puede ser reescrito
como un
sistema mecánico-estadístico, las poderosas herramientas de este campo pueden
ser aplicadas al estudio de sistemas biológicamente relevantes, llevando a
predicciones cualitativas y, en algunos casos, hasta cuantitativas. 

Siguiendo este enfoque, el primer paso y a veces el más importante, es encontrar la mejor
manera de traducir el problema original en un sistema físico cuya función energía
describa todas las interacciones relevantes y las ligaduras del problema,
y al mismo tiempo dicha función energía sea lo bastante simple como para ser convenientemente
tratada con métodos analíticos o numéricos.

En el trabajo que hemos presentado, se ha aplicado esta metodología 
a dos problemas biológicos aparentemente muy distintos: 
en el primero de los dos hemos afrontado el problema de la
secuenciación de péptidos procedentes de Espectrometría  de Masa Tándem,
mientras que en el segundo hemos analizado el plegamiento de la Miotrofina, una
pequeña proteína de tipo modular.

Estos problemas han sido traducidos a un sistema estadístico unidimensional con
interacciones locales a primeros vecinos, o con interacciones que puedan
ser reducidas a bloques: en ambos casos se puede calcular
exactamente la distribución al equilibrio a través de la aplicación de un
calculo de tipo matriz de transferencia. 

\paragraph{Secuenciamiento de Proteínas}
En la primera parte hemos afrontado el problema de la secuenciación de
péptidos que representa un desafío importante y aún sin resolver en el campo
de la Espectrometría de Masa Tándem.
\'Esta es una técnica común, rápida y fiable la cual proporciona un espectro de masa
de un péptido que contiene la información necesaria para poder inferir la
secuencia de los residuos que lo componen.
La Espectrometría de Masa Tándem, gracias a su simplicidad y relativamente bajo
coste, es ampliamente utilizada y generalmente incrustada en una instrumentación
high-throughput automatizada, donde una gran cantidad de péptidos
vienen analizados automáticamente.
Por este motivo la Espectrometría de Masa Tándem permite recolectar una enorme cantidad de espectros que 
necesitan una herramienta automatizada para su interpretación.

Los algoritmos disponibles actualmente son efectivos si aplicados a un número restringido
de casos, como por ejemplo el análisis de proteínas cuya secuencia es conocida y
recolectada en una base de datos, por otra parte éstos se vuelven ineficientes si utilizados en
situaciones no estándar como la caracterización del contenido proteico de
sistemas cuyo proteoma es todavía desconocido o el secuenciación de
proteínas mutadas o modificadas.

Nuestro enfoque frente a este desafío se basa en la traducción del problema de
secuenciación al estudio del comportamiento en el equilibrio de un sistema
físico. Este estudio se basa en el
diseño de un adecuado potencial \emph{ad hoc} derivado del análisis de la
distribución típica de los iones en el plano del espectro, y en el cálculo
exacto de la función de partición y de otras variables termodinámicas relevantes
del sistema, a partir de las cuales se puede calcular la secuencia del precursor.

Nuestro método \ournovo~ introduce una temperatura ficticia como parámetro del
algoritmo.
Este parámetro, como en un sistema termodinámico real, actúa de interruptor
que controla la cantidad de fluctuaciones presentes en el sistema y
distingue dos regímenes:
uno a baja temperatura donde el sistema está atrapado en un mínimo energético
que refleja la información presente en el espectro estudiado, y otro a alta
temperatura donde el sistema puede explorar una parte más grande del espacio de
las configuraciones.
A baja temperatura, nuestro algoritmo se comporta de manera
similar a los demás \emph{de novo} algoritmos prediciendo la secuencia que 
mejor se ajusta a los datos experimentales. A temperaturas más altas, el sistema
explora regiones del espacio de las secuencias alejadas del mínimo energético,
que puede ser útil para evaluar la validez de la predicción. 


El algoritmo ha sido testado mediante una base de datos de espectros
acoplados con una interpretación fiable de la secuencia de los péptidos
originarios, todos doblemente cargados, y sobre la misma base de datos han sido testados
algunos de los algoritmos \emph{de novo} disponibles (NovoHMM, PepNovo, Lutefisk).
Nuestro algoritmo produce resultados comparables con los programas existentes
y además exhibe algunas característica útiles relacionadas con el sistema
termodinámico asociado que no se encuentra en los demás.
Sobresale la posibilidad del algoritmo de controlar la temperatura de trabajo
junto con la posibilidad de calcular exactamente la distribución de
probabilidad al equilibrio, estas características en su conjunto permiten considerar al
mismo tiempo el entero espacio de las secuencias.

Futuros desarrollos incluyen mejoras en el algoritmo final, sobre todo en
la forma de la función coste.
Mejoras en el algoritmo implicaran probablemente una redefinición de la
base de datos de aprendizaje, incluyendo unos filtros más específicos y estrictos y una
definición más detallada de la masa del péptido precursor, que se ha
demostrado fundamental para mejorar las predicciones del modelo.
Una caracterización de la distribución de los iones separada para diferentes
espectrómetros puede mejorar la definición de la función coste, debido
al echo que la física subyacente los diferentes procesos de fragmentación y
de separación puede producir diferentes patrones espectrales.

La natura de este método, basado en la información global incrustada en el espacio de las
configuraciones sintetizada en el perfil de probabilidad de fragmentación, puede ser acoplado a un
algoritmo clásico de búsqueda sobre base de datos.
Este último usualmente busca el péptido más probable en una base de datos de péptidos
según una propia función coste. El potencial \emph{de novo} descrito en este trabajo puede ser
aplicado a las secuencias elegidas desde la base de datos como una función coste
refinada, creando una nueva puntuación basada en las distribuciones de los señales
y del ruido del espectro considerado.

\paragraph{Plegamiento de Proteínas}

En la segunda parte de esta Tesis hemos aplicado el modelo para el plegamiento
de proteínas WSME para describir el comportamiento de la Miotrofina.
Esta proteína presenta una estructura en módulos, cada uno de los cuales está formado
por dos hélices antiparalelas con cierta estabilidad intrínseca. 
Los  cuatro módulos que componen la molécula se disponen paralelos entre sí formando una estructura lineal.
Estas proteínas muestran experimentalmente un plegamiento cooperativo típico de
las proteínas globulares que parece 
en contraste con la modularidad estructural, atrayendo la atención
de los investigadores. 

Nuestros resultados reproducen cualitativamente el comportamiento experimental
de la molécula.
El modelo aplicado al análisis del equilibrio y de la cinética del sistema,
reproduce la cooperación en ambos ámbitos, y una investigación más profunda 
muestra que dicha cooperación es compatible con un
paisaje de energía libre con múltiples mínimos, predichos por el modelo.
La simulación del la cinética del proceso de plegamiento revela la presencia de
una heterogeneidad de caminos, como propuesto por \citet{Lowe2007a} para
explicar el comportamiento de la proteína wild type y de sus
mutantes en un único marco.
Además, el modelo ha sido aplicado al análisis de algunos mutantes
correspondientes a la aplicación de perturbaciones locales y no locales, permitiendo
la predicción cualitativa de comportamientos experimentales (en particular, se
muestra que las mutaciones actúan como un interruptor en el flujo 
hacía un camino u otro de plegamiento y desnaturalización).

Una modificación del modelo para conseguir resultados cuantitativos puede ser
introducida ajustando los parámetros del modelo a los datos experimentales
disponibles, como mostrado por \citet{Bruscolini2011}.
Los parámetros son adaptados para reproducir los valores de $C_P$ a través del
uso de expresiones fenomenológicas derivadas del mismo modelo WSME.
Se ha mostrado que este modelo describe con éxito y cuantitativamente el plegamiento de una
proteína de tipo down-hill y una proteína a dos estados\cite{Bruscolini2011}.
Desafortunadamente, este refinamiento del modelo no puede ser aplicado al caso
de la Miotrofina en cuanto se basa en un ajuste a datos calorimétricos que no están
disponibles para ella.

En este marco un acercamiento interesante puede ser el estudio de proteínas
modulares similares, compuestas por módulos idénticos, las llamadas proteínas
modulares de consenso.
Estas proteínas están compuestas por repeticiones de una misma cadena de
residuos; éstas, aunque non sean proteínas naturales, han sido sintetizadas en
laboratorio y sus estructuras han sido resueltas.
El estudio de las proteínas de consenso puede aportar a una comprensión mayor de
los procesos que gobiernan la dinámica microscópica en los polímeros modulares.

Concluyendo, de acuerdo con las observaciones anteriores, creemos que el uso de
simples modelos mecánico-estadísticos nos permitirá producir un cantidad
considerable de resultados cualitativos y hasta cuantitativos 
en problemas de inspiración biológica, en el próximo futuro.
